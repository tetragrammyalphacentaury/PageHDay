\color{Orange}
    Одним из самых перспективных направлений в гомоморфном шифрованиии является класс NTRU-систем, также использующих решетки. В 2009 году NTRU упоминается в работе Джентри \cite[p.65]{Gentry-09}, как первая криптосистема, использующая структуры идеалов на решетках. Первая работа по NTRU  \cite{Hoffstein-98} была опубликована в 1988 году Хоффштейном, Пифером  и Сильверманом и изначально подразумевала собой криптосистему с публичным ключом на кольцах, которая лишь в дальнейших работах получила развитие и связь с решетками (1997-2001 год) \cite{Coppersmith-97,May-99,Gentry-01}. В дальнейшем NTRU-криптография строится вокруг работы Миклоша Айтая \cite{Ajtai-97}, на его принципе эквивалента среднего/худшего; цепочка работ, развивающих проблематику NTRU \cite{Micciancio-02,Peikert-06,,Peikert-07,Lyubashevsky-10,Micciancio-11} вышла в период с 2002 по 2007 годы. Параллельно работы Хоффштейна \cite{Hoffstein-98,Hoffstein-05} представили алгоритмы шифрования и электронной подписи, криптографическая стойкость которых была исследована в работе \cite{Stehlé-11}.\par
    Дальнейшее развитие NTRU получило с выходом серий работ, представляющих системы NTRU-LWE и NTRU Prime \cite{Bernstein-16}.\par
    Несмотря на то, что с момента появления NTRU-шифров прошло более двух десятилетий, NTRU-схема получила внимания лишь после открытия полностью гомоморфного шифрования и возросшего интереса к криптографии на решетках. Оба вопроса являются перспективными для NTRU, что делает его актуальным направлением в современной криптографии, особенно, учитывая что NTRU обладает хорошей асимптотической производительностью и малым размером шифртекста. \cite{Shor-99,HoffsteinLatticeCrypto-09}\par
\color{Blue}
    Возможность существования полностью гомоморфного шифрования на NTRU было впервые показано в \cite{Lopez-12} и \cite{Gentry-12}. Помимо проблемы решеток, система \cite{Lopez-12} также строится на мало изученной проблеме Decisional Small Polynomial Ratio (DSPR). В \cite{Bos-13} схема была избавлена от DSPR. Последовательно \cite{Brakerski-12} показал технику тензорирования, с помощью которой можно ограничить рост ошибки при гомоморфных операциях и также избавиться от DSRP. Однако, это техника требует большого размера ключа вычислений и комплексность в протоколе при ключевом переключении, что делает схему непрактичной. Все схемы, которые пытаются уйти от DSRP уязвимы к определенному виду атак. В 2016 году [Doroz-16] появилась модицифированная схема FHE для NTRU, не использующая DSRP и, кроме этого, не требующая ключ вычислений при произведении гомоморфных операций, что делает схему очень привлекательной для исследователей. Вместо этого она использует технику выравнимания шума \cite{Stehle-11}, которая была получена из схемы Джентри \cite{Jentry-13}.\par
    Актуальными направлениями для NTRU на данном этапе является дальнейшее получение практической FHE схемы, что является критически необходимым шагом, а также реализация вычислительного потенциала за счет оптимальной аппаратной реализации [Doroz-14,Dai-14,LiuWu-15]. Перспективной является также предложенная в 2014 году схема [Rohloff-14], где используются элементы самонастройки [Alperin-13] и "double-CRT" [Gentry-12] для преобразования шифртекстов в соответствии с текущей задачей.\par
\normalcolor
