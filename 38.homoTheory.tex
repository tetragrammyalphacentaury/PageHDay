\color{Magenta}  
    Первой работой по гомоморфному шифрованию принято считать [RivestDataBanks-78]. Развитие описанные идеи получили в работе Брикеля и Якоби ``On privacy homomorphisms''. В течении последующих 30 лет удавалось получить лишь частичные результаты, то есть такие системы, где поддерживалось бы гомоморфное либо сложение, либо умножение, но ни обе операции вместе. Такие системы носят название, соответственно, частично гомоморфных систем (PHE). [AsurveyOnEncSchemes].

    В 1980-х со становлением вероятностной криптографии формируются требования к гомоморфным системам: способность выполнять любые операции (под любыми обычно понимаются умножение и сложение) в неограниченном количестве, обладать семантической защищенностью. Также должно выполнятся требование на ограниченное увеличение размера шифртекста после каждой операции, что косвенно отражает свойство отсутствия  ограничений на количество выполнений.

    Далее, в 1990-х теория алгоритмов позволила создать удобные представления для вычислений. Модель вычислений теперь может задаваться не только формулой, но и графом, таблицей истинности, булевой функцией, логической цепью, конечным автоматом и т. д. С 2000-х годов начинают развиваться почти гомоморфные системы  (Somewhat Homomorfic Encryption) -- это такие системы, которые могут выполнять как операции сложения, так и операции умножения, но в ограниченном количестве.

    Затем, начинают затрагиваться все уровни криптографической структуры, приводящее, таким образом, к тому, что Джентри синтезирует полностью гомоморфную систему, развитие которой вылилось в три поколения. В процессе развития этой системы, появилась возможность изучить особые свойства гомоморфных систем, появилась более глубокая теоретическая база. Кроме этого, появилось направление, которая уделяет особое внимание функциям вычислений -- обфускация. На данный момент гомоморфные криптосистемы имеют эффективные алгоритмы, а вектор их равития лежит в направлении интегрирования и дальнейшего усложнения.

    За все время была выработана следующая классификация для гомоморфного шифрования: частично-гомоморфное, ограниченно-гомоморфное и полностью гомоморфное шифрование.

\subsection{Частичное гомоморфное шифрование}

    Список частично-гомоморфных систем представлен ниже:
    \tiny
    \begin{longtable}{|p{0.1in}|p{0.3in}|p{0.7in}|p{0.2in}|p{0.2in}|p{0.9in}|p{0.7in}|p{1.3in}|}\hline 
        \#  & Год & Название & \multicolumn{2}{|p{0.4in}|}{Операции} & Вычислительная проблема & Улучшение какой системы & Примитив, свойства \\ \hline 
            & \multicolumn{7}{|p{4.3in}|}{Частично гомоморфные системы} \\ \hline 
        1   & 1978 & Система RSA &  & * & Факторизация [Montgomery-94] &  &  \\ \hline 
        2   & 1982 & Система Гольдвассер-Микали & ? &  & Проблема квадратичных вычетоа (Quadratic Residuosity Problem) [Kalinski-2005] &  & Вероятностная криптосистема, зашифровывает побитно \\ \hline 
        3   & 1985 & Система Эль-Гамаля &  & * & Дискретное логарифмирование [Kevin-90] & Система RSA &  \\ \hline 
        4   & 1994 & Система Бенало & + &  & Вычеты произвольной степени (Higher Residuosity Problem) [BenalohRooting-87] & Система Гольдвассер-Микали & Вероятностная криптосистема, зашифровывает блок данных в виде полинома \\ \hline 
            & 1998 & Система Накаша-Штерна & + &  & Вычеты произвольной степени (Higher Residuosity Problem) & Система Бенало & Улучшение производительности за счет изменения схемы расшифрования \\ \hline 
            & 1998 & Система Окамото-Утиямы & + &  & Квадратичные вычеты, факторизация & Система RSA, система Гольдвассер-Микали & Вероятностная криптосистема, улучшение производительности за счет использования других множеств \\ \hline 
        5   & 1999 & Система Пэйе & + & К & Комплексные вычеты (Сomposite Residuosity Problem) [Jager-12] &  & Можно добавить гомоморфное умножение, если знаешь открытый текст одного из сообщений, умножение на скаляр, вероятностная криптосистема \\ \hline 
            & 2001 & Система Дамгода-Джурика & + &  &  & Система Пэйе & Вероятностная криптосистема \\ \hline 
            & 2002 & Система Гэлбрейта & + &  &  & Система Пэйе & Эллиптические кривые \\ \hline 
            & 2007 & Система Кавачи & + &  & Поиск решения на решетках &  & Большая циклическая группа, решетки, псевдогомоморфизм \\ \hline 
    \end{longtable}
    \normalsize
\normalcolor
    Можно выделить следующие вычислительные проблемы, на которых может быть построен частичный гомоморфизм:

    \begin{enumerate}
        \item  Факторизация
        \item  Квадратичные вычеты, вычеты произвольной степени
        \item  Композитные вычеты
        \item  Решетки
        \item  Дискретное логарифмирования
        \item  Линейные коды
    \end{enumerate}
\normalcolor

\subsection{Ограниченно-гомоморфное шифрование}

\color{RoyalBlue}
    Отдельные механизмы были выработаны в классе ограниченно-гомоморфных систем. Особенностью этого класса является возможность конструирования функции, элементарные же функции состоят из базиса булевой алгебры.

    Для этого класса характерно использование бинарных таблиц и забывчивой передачи, как основных элементов при построении системы. Проблемы, которые необходимо решить при этом - это ограничение на рост размера шифртекста, а также реализация устойчивого протокола с фиксированным количеством раундов. В первой такой системе, которая припысывается Яо [Yao-82], участники общаются каждый раунд и узнают, нужна ли помощь в формировании выходного значения до тех пор, пока не будет пройдена вся цепочка вычислений. В этом случае глубина вычислений -- основной фактор, который влияет на комплексность криптосистемы.

    Система Сендера [Sender-99] является развитием системы Яо; в его системе в качестве входного значения используется полином, вычисляющийся с использованием NC-цепей, поэтому все операции происходят за один раунд. Однако, в этом случае размер шифртекста растет экспоненциально, так как основной ограничивающий фактор не глубина вычислений, а рамер бинарной таблицы. 

    Качественный скачок в развитии представляет система Бонеха-Го-Ниссима [Boneh-05], которая вычисляя 2-DNF-формулы над шифртекстом, обеспечивает как алгебраический набор операций, так и константный размер шифртекста.

    Система Ишая-Пашкина расширяет область гомоморфных вычислений на ациклические графы принятия решений, более генерализированным множеством, чем таблица истинности.

    Особенностью вышеперечисленных систем является использование бинарных операций вместо алгебраических, что не позволяет им полностью соответствовать классу алгебраически гомоморфных систем. Таким системы, также, являются особым случаем для теоремы Бонех и Липтона, которые показали, что детерминированные алгебраически гомоморфные системы над кольцами ${Z}/{NZ}$ могут быть сломаны за время не выражающееся экспоненциальной зависимостью. Но для систем ${Z}/{2}Z$ необходимо выполнение условия вероятностной системы, если они реализуют гомоморфные вычисления.

    \tiny
    \begin{longtable}{|p{0.1in}|p{0.3in}|p{0.6in}|p{0.2in}|p{0.3in}|p{0.7in}|p{0.8in}|p{1.3in}|} \hline 
    \#  & Год & Название & \multicolumn{2}{|p{0.5in}|}{Операции} & Математический примитив & Криптографический примитив & Размер шифртекста \\ \hline 
    1   & 1982 & Система Яо [Yao-82] & AND & OR & ${Z}/{2}Z$ & Искаженная схема (garbled circuit), забывчивая передача (oblivious transfer) & Растет линейно с каждой элементарной операцией; переменное количество раундов в протоколе, которая зависит от глубины вычислений \\ \hline 
    2   & 1994 & Система Феллоуза-Коблица, ``Polly Cracker'' & + & x &  &  & Размер шифртекста растет экспоненциально после каждой операции \\ \hline 
    3   & 1999 & Система Сендера [Sender-99] & AND & 1-OR\newline или\newline 1-NOT & ${Z}/{2}Z\left[x\right]$ & $NC^1$-цепи, забывчивая передача & Шифртекст растет экспоненциально, гомоморфизм на основе полугруппы, один раунд в протоколе \\ \hline 
    4   & 2005 & Система Бонеха-Го-Ниссима & + & 1-x & ${Z}/{2}Z\left[x\right]$ & 2-DNF формулы & Проблема подмножеств [Gjosteen-04], шифртекст имеет константный размер \\ \hline 
    5   & 2007 & Система Ишая-Пашкина & + &  &  & Ациклический граф (вычисление ветвлений, binary desicion diagrams) & Вероятностная криптосистема, не зависит от размера функции \\ \hline 
    \end{longtable}
    \normalsize
\normalcolor

    Отдельного внимания заслуживает система Мельчора [Melchor-10], которая опубликовалась после работы Джентри и которая разработала способ цепного шифрования, где каждое звено для этого шифрования может быть сформировано на основе примитива из другой существующей криптосистемы. Цепное шифрование позволяет производить гомоморфные вычисления заданной глубины, которая зависит от количества используемых примитивов их свойств, от чего также зависит наличие набора алгебраических операций.

\subsection{Полностью гомоморфное шифрованияе}

\color{Blue}
    Система Джентри в качестве своей основы использует решетки \cite{Jentry-09}, которые также получили огромное внимание после его работы. Решетки признаются устойчивыми к квантовому анализу \cite{Regev-06}, кроме этого они имеют достаточно обширный теоретический фундамент, поэтому их использование можно отнести к достоинствам системы. Теория решеток впервые была опубликована в \cite{Minkowski-68}, с тех пор было разработано несколько достаточно стойких вычислительных задач, наиболее используемыми из которых являются задачи поиска ближайшего  и кратчайшего вектора \cite{Peikert-15} и проблема среднего/худшего \cite{Atjai-96}. В \cite{Goldreich-97} выл представлен способ вычислительной редукции решеток, что напрямую связало их с теорией криптографии. Решетки могут комбинироваться с другими математическими примитивами, что определяет различные классы таких систем.\par

    \vspace{8mm}Так, например, система Джентри относится к классу систем, построенных на решетках идеалов. Подобное решение позволило реализовать ассиметриченую гомоморфную систему шифрования, то есть систему с публичным ключом \cite{Hoffstein-98}. Однако, она довольно сложна в реализации и имеет некоторые недостатки в плане производительности, особенно это касается перешифровки шифртекста. За последние годы было предложено много способов ее оптимизации. В 2010 году предложено улучшение схемы генерации ключа, а также улучшение стойкости гомоморфного шифрования \cite{Jentry-10}. Вариант схемы Джентри, работающий на шифртексте и ключе меньшего размера без потери стойкости был представлен в \cite{SmartVercauteren-10}. Поздние работы направлены на дальнешее улучшение алгоритма генерации ключа и также алгоритма "перешифровки" шифртекста. Также разработана ограничено-гомоморфная схема с пространством открытого текста большей мощности, что увеличивает количество гомоморфных операций \cite{Mikus-12}\par

   \vspace{8mm}Существует класс систем на решетках, которые используют проблему LWE \cite{Regev-09} и ее алгебраический вариант - ring-LWE \cite{Lyubashevsky-13}. Эти проблемы признаются наиболее стойкими, так как они позволяют использовать меньший размер шифртекста без потери защищенности. В 2011 году была предложена схема ограниченного гомоморфного шифрования \cite{BrakevskiVaikuntanathan-11} на основе RLWE, где была показана большая производительность, чем в LWE. Эта схема в той же работе была дополнена до полностью гомоморфной схемы.\par
    Все гомоморнфые схемы появившиеся после этой работы принадлежат к категории систем второго поколения, например, система использующая технику перелинеаризации \cite{BrakevskiVaikuntanathan-14}, которая устанавливает стабильный размер для шифртекста большого размера и обходится без процедуры перешифровки шифртекста. В качестве тенденций развития можно обозначить уровневые системы полностью гомоморфного шифрования, которые повышают производительность за счет использования функций с ограниченной глубиной вычислений (ограниченным набором элементарных операций) \cite{Brakerski-14}, линейный рост ошибки с каждой операцией \cite{Peikert-15}, а также системы с шифрованием атрибутов (идентификацией, множественными ключами) и собственными векторами \cite{Gentry-13}\par
    Актуальной схемой для доработки полностью гомоморфногом шифрования является схема \cite{Cheon-16}, которая использует проблему аппроксимации общего делителя, поддерживая большую мощность пространства сообщений.\par

    \vspace{8mm}\color{Orange}
    Одним из самых перспективных направлений в гомоморфном шифрованиии является класс NTRU-систем, также использующих решетки. В 2009 году NTRU упоминается в работе Джентри \cite[p.65]{Gentry-09}, как первая криптосистема, использующая структуры идеалов на решетках. Первая работа по NTRU  \cite{Hoffstein-98} была опубликована в 1988 году Хоффштейном, Пифером  и Сильверманом и изначально подразумевала собой криптосистему с публичным ключом на кольцах, которая лишь в дальнейших работах получила развитие и связь с решетками (1997-2001 год) \cite{Coppersmith-97,May-99,Gentry-01}. В дальнейшем NTRU-криптография строится вокруг работы Миклоша Айтая \cite{Ajtai-97}, на его принципе эквивалента среднего/худшего; цепочка работ, развивающих проблематику NTRU \cite{Micciancio-02,Peikert-06,,Peikert-07,Lyubashevsky-10,Micciancio-11} вышла в период с 2002 по 2007 годы. Параллельно работы Хоффштейна \cite{Hoffstein-98,Hoffstein-05} представили алгоритмы шифрования и электронной подписи, криптографическая стойкость которых была исследована в работе \cite{Stehlé-11}.\par
    Дальнейшее развитие NTRU получило с выходом серий работ, представляющих системы NTRU-LWE и NTRU Prime \cite{Bernstein-16}.\par
    Несмотря на то, что с момента появления NTRU-шифров прошло более двух десятилетий, NTRU-схема получила внимания лишь после открытия полностью гомоморфного шифрования и возросшего интереса к криптографии на решетках. Оба вопроса являются перспективными для NTRU, что делает его актуальным направлением в современной криптографии, особенно, учитывая что NTRU обладает хорошей асимптотической производительностью и малым размером шифртекста. \cite{Shor-99,HoffsteinLatticeCrypto-09}\par
\color{Blue}
    Возможность существования полностью гомоморфного шифрования на NTRU было впервые показано в \cite{Lopez-12} и \cite{Gentry-12}. Помимо проблемы решеток, система \cite{Lopez-12} также строится на мало изученной проблеме Decisional Small Polynomial Ratio (DSPR). В \cite{Bos-13} схема была избавлена от DSPR. Последовательно \cite{Brakerski-12} показал технику тензорирования, с помощью которой можно ограничить рост ошибки при гомоморфных операциях и также избавиться от DSRP. Однако, это техника требует большого размера ключа вычислений и комплексность в протоколе при ключевом переключении, что делает схему непрактичной. Все схемы, которые пытаются уйти от DSRP уязвимы к определенному виду атак. В 2016 году [Doroz-16] появилась модицифированная схема FHE для NTRU, не использующая DSRP и, кроме этого, не требующая ключ вычислений при произведении гомоморфных операций, что делает схему очень привлекательной для исследователей. Вместо этого она использует технику выравнимания шума \cite{Stehle-11}, которая была получена из схемы Джентри \cite{Jentry-13}.\par
    Актуальными направлениями для NTRU на данном этапе является дальнейшее получение практической FHE схемы, что является критически необходимым шагом, а также реализация вычислительного потенциала за счет оптимальной аппаратной реализации [Doroz-14,Dai-14,LiuWu-15]. Перспективной является также предложенная в 2014 году схема [Rohloff-14], где используются элементы самонастройки [Alperin-13] и "double-CRT" [Gentry-12] для преобразования шифртекстов в соответствии с текущей задачей.\par
\normalcolor


    \vspace{8mm}Одним из вариантов, не связанных с решетками, является схема, предложенная [VanDijk-10]. Эта схема использует ограниченно-гомоморфную схему, построенную на целых числах и модульной арифметике, которая затем использует метод Джентри для получения полностью гомоморфной схемы за счет "самонастройки" (bootstrapping).\par
Вычислительная сложность системы базируется на задаче аппроксимации поиска наибольшего общего делителя [Galbraith-16].\par
На данный момент реализована симметричная и ассиметричная гомоморфная система на целых числах; особенностью данной схемы является простота в реализации, взамен схема обладает низкой вычислительной способностью.\par
Основные направления развития данного класса систем включают уменьшение размера публичного ключа [Coron-11] [Coron-12] [Yang-12], а также улучшение алгоритмов генерации ключей [RamaiahKumari-12] и перешифровки [Chen-14]. Также существует версия с упаковкой шифртекстов [Cheon-13]\par
На данный момент существует множество подходов к улучшению системы на целых числах: масштбируемое инвариативное полностью гомомофное шифрование [Coron-14], схема с открытым текстом в виде целых чисел [RamaiahKumari-12], ограниченно-гомоморфная система с арифметикой больших чисел [Pisa-12], полностью гомоморфная схема без самонастройки [Aggarwal-14], а также схема в небинарном пространстве сообщений [NuidaKurosava-15].


\normalcolor
