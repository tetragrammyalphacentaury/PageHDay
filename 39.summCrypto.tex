    Подведём итоги развития гомоморфного шифрования и выделим актуальные направления развития.
    \begin{enumerate}
        \item Вычислительные проблемы
		\begin{enumerate}
			\item Модульная арифметика
		            \begin{enumerate}
                              \item Факторизация
                              \item Квадратичные вычеты
                              \item Композитные вычеты
                              \item Вычеты произвольной степени
                              \item Дискретное логарифмирование
                              \item Поиск наибольшего делителя
                        \end{enumerate}
			\item Решетки
		            \begin{enumerate}
			            \item Проблема проблема эквивалента среднего/худшего базиса
			            \item Прооблема кратчайшего вектора
			            \item Аппроксимация среднего вектора
                              \item Проблема соседнего вектора
                        \end{enumerate}
			\item Обучение с ошибками
		            \begin{enumerate}
			            \item Обучение с ошибками в пространстве колец
                        \end{enumerate}
			\item Комбинаторные проблемы
		            \begin{enumerate}
                              \item Укладка рюкзака
			            \item Subset-sum
                              \item Раскраска графа
                        \end{enumerate}
		\end{enumerate}
	\item Математические примитивы
		\begin{enumerate}
			\item Бинарная логика
			\item Пространство остатков целых чисел 
			\item Идеалы
			\item Усеченые кольца
			\item Линейные коды
                  \item Биллинейные спаривания на эллиптических кривых
		\end{enumerate}
        \item Криптографические примитивы
		\begin{enumerate}      
                  \item Atomic proxy cryptography
		      \begin{enumerate}
                        \item Самонастройка (bootstraping)
			      \item Перешифровка (refleshing)
			      \item Упаковка шифртекста
			      \item Пороговое ограничение шифртекста (перелинеаризация)
		      \end{enumerate}
                  \item Односторонняя функция
                  \item Забывчивая передача
                  \item Защищенные индексы
		\end{enumerate}
	\item Криптографические схемы
		\begin{enumerate}
	            \item Вычисление функций
		            \begin{enumerate}
	                        \item Искаженная схема
	                        \item $NC^{1}$-цепи
                              \item Формулы 2-DNF
                              \item Ациклические графы
		            \end{enumerate}
	            \item Пороговое шифрование
		            \begin{enumerate}
	                        \item Шифрование атрибутов
	                        \item Выборочная идентификация
		            \end{enumerate}
	            \item Гомоморфность
		            \begin{enumerate}
	                        \item Частичная
	                        \item Ограниченная
                              \item Полная
		            \end{enumerate}
	            \item Обфускация
	            \item Арифметика больших чисел
	            \item Публичный ключ
		\end{enumerate}
    \end{enumerate}


