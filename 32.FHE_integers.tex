Одним из вариантов, не связанных с решетками, является схема, предложенная [VanDijk-10]. Эта схема использует ограниченно-гомоморфную схему, построенную на целых числах и модульной арифметике, которая затем использует метод Джентри для получения полностью гомоморфной схемы за счет "самонастройки" (bootstrapping).\par
Вычислительная сложность системы базируется на задаче аппроксимации поиска наибольшего общего делителя [Galbraith-16].\par
На данный момент реализована симметричная и ассиметричная гомоморфная система на целых числах; особенностью данной схемы является простота в реализации, взамен схема обладает низкой вычислительной способностью.\par
Основные направления развития данного класса систем включают уменьшение размера публичного ключа [Coron-11] [Coron-12] [Yang-12], а также улучшение алгоритмов генерации ключей [RamaiahKumari-12] и перешифровки [Chen-14]. Также существует версия с упаковкой шифртекстов [Cheon-13]\par
На данный момент существует множество подходов к улучшению системы на целых числах: масштбируемое инвариативное полностью гомомофное шифрование [Coron-14], схема с открытым текстом в виде целых чисел [RamaiahKumari-12], ограниченно-гомоморфная система с арифметикой больших чисел [Pisa-12], полностью гомоморфная схема без самонастройки [Aggarwal-14], а также схема в небинарном пространстве сообщений [NuidaKurosava-15].
