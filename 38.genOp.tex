
\newcommand{\cmm}{\textrm{,}}

      В криптографической системе можно выделить  несколько областей, состоящих из различных уровней. Это некоторые математические примитивы, которые сформированы вокруг сложной вычислительной проблемы, а также криптографические примитивы, которые организуют уровень криптографической схемы.

\color{Gray} \tiny
      Для примера выделим обозначенные уровни в криптосистеме RSA. Так, уровень вычислительной проблемы представлен задачей факторизации целых чисел. Уровень математических примитивов представлен операциями возведения в степень по модулю в пространстве целых чисел. Уровень криптографических примитивов составляет такой примитив, как односторонняя функция. Наконец, последний уровень характеризуется протоколом обмена ключами и ассиметричностью (наличием, публичного и секретного ключа).
\normalcolor \normalsize

      Условно, что в такой схеме можно провести следующую границу: первые два уровня формируют область математической абстракции, в то время, как последние два уровня формируют область преобразования информации и взаимодействия с ней.

\color{Gray} \tiny
      Подобное разделение позволяет описать системы с хэшированием, шифрованием, цифровой подписью или другими соответствующими информационными процессами. Однако, для гомоморфных систем имеет место следующее уточнение.
\normalcolor \normalsize

      Гомоморфные свойства можно обозначить, как продукт, принадлежащий одновременно уровнями математических и криптографических примитивов, что означает, что их можно отделить как от низлежащей математической проблемы, так и от криптографической схемы, т. е. выделить уровень вычислительной абстракции.

      Попытаемся показать это через определение обобщенных операций. Определим такую операцию над шифртекстом. Известно, что в общем случае операция над шифртекстом может определяться через последовательность некоторых элементарных операций, то есть представлять собой выполнение алгоритма. Такие алгоритмические функции, обозначаемые словами, например, Add и Mult, вызывают соответствующие преобразования со стороны открытого текста. Определив множество элементов открытого текста $c\in C$ и шифртекста $\psi \in \Psi $, мы можем записать, например:

      \[Add\left({\psi }_1 \cmm {\psi }_2\right)\to c_1+c_2;~Mult\left({\psi }_1 \cmm {\psi }_2\right)\to с_1+с_2;\] 

      Набор операций сложения и умножения образует гомоморфную систему шифрования. Обобщая бинарную операцию с множества открытого текста на множество шифртекста, мы можем выделить следующие свойства:

      \begin{enumerate}
            \item  Множество открытого текста является подмножеством шифртекста, $C\in \Psi $, и оба множества могут быть разбиты  с помощью элементов одних и тех же непересекающихся множеств, образующих множество $R$. Рассматривая элементы $c\in C$ как элементы группы, элементы $\psi \in \Psi $ являются векторами над элементами множества $C$. 

            \[C\in R:\Psi \in R^k;R=R_1\cup R_2\cup R_3\dots \cup R_N;~N=\left|R\right|;\] 

            \item \textit{ }Операции над шифртекстом и открытым текстом можно определить через одинаковые операции над множеством $R$.

            \item  При этом результат операций над шифртекстом оставляет возможность расшифрования и получения октрытого текста на уровне криптографической схемы.
      \end{enumerate}

      Тогда операция $Op$ является обобщенной, если:

            \[Op:~{\psi }_1\odot {\psi }_2\to c_1\odot c_2;\] 

            \[a \cmm b\in C:a\odot b:\forall R_i\subset R\bigcap C \cmm \ r^i\in R_i:\left[r^1_a\odot r^1_b\ ;r^2_a\odot r^2_b;r^3_a\odot r^3_b;\dots \right];\] 

\color{Gray} \tiny
      Например, конгруэнтная криптосистема поддерживает обобщенную гомомофрную операцию сложения по умолчанию, благодаря преобразованиям на множестве идеалов.
\normalcolor \normalsize

      На уровне множеств любая криптосистема может быть дополнения обобщенными операциями без нарушения своей структуры до уровня криптографической схемы. На последнем же уровне необходимо разрешить вопрос расшифрования с учетом накапливаемой ошибки. При этом не всякая криптосисема может это позволить.
